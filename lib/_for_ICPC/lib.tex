\documentclass[11pt,a4paper]{jsarticle}
\usepackage[top=20truemm,bottom=20truemm,
			left=25truemm,right=20truemm]{geometry}
\usepackage{url,ascmac,framed,listings,jlisting}
\usepackage[usenames]{color}
\usepackage[dvipdfmx]{graphicx}
\usepackage{booktabs}
\usepackage{multirow}

%ここからソースコードの表示に関する設定
\lstset{
    language=c++,
    showstringspaces=false,
    frame=single,
    frameround=tttt,
    basicstyle={\ttfamily\small},
    numbers=left,
    numberstyle={\ttfamily\small},
    numbersep=10pt,
    tabsize=2,
    extendedchars=true,
    xleftmargin=17pt,
    framexleftmargin=17pt,
}
\definecolor{shadecolor}{gray}{0.85}
\definecolor{linecolor}{gray}{0.55}

\makeatletter
    \newenvironment{myproblem}
    {\vspace{1zh}\begin{itembox}[c]}
    {\end{itembox}\vspace{1zh}}
    \newenvironment{myoutput}
    {\begin{leftbar}\begin{lstlisting}}
    {\end{lstlisting}\end{leftbar}}
\renewenvironment{leftbar}{%
%  \def\FrameCommand{\vrule width 3pt \hspace{10pt}}%  デフォルトの線の太さは3pt
  \def\FrameCommand{\textcolor{linecolor}{\vrule width 1pt} \hspace{10pt}}%
  \MakeFramed {\advance\hsize-\width \FrameRestore}}%
 {\endMakeFramed}
\makeatother
\renewcommand{\labelitemi}{・}


% 章番号の書式の定義
\renewcommand{\thesection}{\arabic{section}}

% 横線付きsection
\newcommand{\linesection}[1]{\newpage \section{#1} \hrule width 150mm \vspace{3em}}

% ちょっと強調したい時
\newcommand{\minititle}[1]{\medskip{\large \sf #1}\medskip}

\begin{document}

\section{データ構造}
\minititle{Union-Find}
\begin{lstlisting}
struct UnionFind{
    vector<int> dat;

    UnionFind(int sz){
        dat.assign(sz, -1);
    }

    bool unite(int x, int y){
        x = root(x), y = root(y);
        if(x == y) return(false);
        if(dat[x] > dat[y]) swap(x, y);
        dat[x] += dat[y];
        dat[y] = x;
        return(true);
    }

    int root(int k){
        if(dat[k] < 0) return(k);
        return(dat[k] = root(dat[k]));
    }

    int size(int k){
        return(-dat[root(k)]);
    }
};

\end{lstlisting}

\minititle{Binary Indexed Tree}
\begin{lstlisting}
template< class T >
struct BinaryIndexedTree {
    vector< T > dat;

    BinaryIndexedTree(int sz) {
        dat.assign(++sz, 0);
    }

    T sum(int k){
        T ret = 0;
        for(++k; k > 0; k -= k & -k) ret += data[k];
        return ret;
    }

    void add(int k, T x){
        for(++k; k < data.size(); k += k & -k) data[k] += x;
    }
};
\end{lstlisting}

\newpage
\minititle{Segment Tree}
\begin{lstlisting}
/*
!) 0-indexed
!) [a, b)に対する演算
!) merge, updateNodeを書く
*/
template<typename Monoid>
struct Segtree{
    int n;
    vector<Monoid> dat;
    Monoid m0; // データの初期化値
    Segtree(int sz, Monoid m0) : m0(m0){
        n = 1;
        while(n < sz) n *= 2;
        dat.assign(2*n-1, m0);
    }

    Monoid merge(Monoid a, Monoid b) // 区間をマージする二項演算
    void updateNode(int k, Monoid x) // 区間を操作する二項演算

    void update(int k, Monoid x){
        k += n-1;
        updateNode(k, x);
        while(k > 0) {
            k = (k-1)/2;
            dat[k] = merge(dat[k*2+1], dat[k*2+2]);
        }
    }
    Monoid query(int a, int b){
        Monoid L = m0, R = m0;
        int A = a+n-1;
        int B = b+n-1;
        while(A < B) {
            if((A&1) == 0) L = merge(L, dat[A++]);
            if((B&1) == 0) R = merge(dat[--B], R);
            A >>= 1;
            B >>= 1;
        }
        return merge(L, R);
    }
    Monoid operator[](const int &k) const { return dat[k+n-1]; }
};

//########## 例 ##########
// Range min (AOJ DSL_2_A)
Monoid merge(Monoid a, Monoid b){ return min(a,b); }
void updateNode(int k, Monoid x){ dat[k] = x; }
Segtree<LL> a(n, (1LL<<31)-1);

// Range add (AOJ DSL_2_B)
Monoid merge(Monoid a, Monoid b){ return a + b; }
void updateNode(int k, Monoid x){ dat[k] += x; }
Segtree<LL> a(n, 0);
\end{lstlisting}

\newpage
\minititle{遅延伝搬 Segment Tree}
\begin{lstlisting}
/*
!) 0-indexed 遅延伝搬セグメント木
!) [a, b)に対する演算
!) M0, L0, merge, updateNode, propagateを書く
*/
template <typename Monoid>
struct LazySegtree {
    int n;
    vector<Monoid> dat, lazy;

    Monoid M0, L0 // データと遅延配列の初期化値 queryに合わせて選択する

    LazySegtree(int sz, Monoid dat_init){
        n = 1;
        while (n < sz) n *= 2;
        dat.assign(2*n-1, dat_init);
        lazy.assign(2*n-1, L0);
    }

    Monoid merge(Monoid a, Monoid b) // 区間をマージする二項演算
    void updateNode(int k, Monoid x) // 区間を操作する二項演算
    void propagate(int k, int l, int r) // 遅延配列の伝搬のさせ方

    void eval(int k, int l, int r) {
        if(lazy[k] == L0) return;
        propagate(k, l, r);
        if(r-l > 1) {
            updateNode(2*k+1, lazy[k]);
            updateNode(2*k+2, lazy[k]);
        }
        lazy[k] = L0;
    }

    void update(int a, int b, Monoid x, int k, int l, int r) {
        eval(k, l, r);
        if (r <= a || b <= l) return;
        if (a <= l && r <= b) {
            updateNode(k, x);
            eval(k, l, r);
        }else{
            update(a, b, x, k*2+1, l, (l+r)/2);
            update(a, b, x, k*2+2, (l+r)/2, r);
            dat[k] = merge(dat[2*k+1], dat[2*k+2]);
        }
    }

    void update(int a, int b, Monoid x) {
        update(a, b, x, 0, 0, n);
    }

    Monoid query(int a, int b, int k, int l, int r) {
        eval(k, l, r);
        if (r <= a || b <= l) return M0;
        if (a <= l && r <= b) return dat[k];
        Monoid L = query(a, b, k*2+1, l, (l+r)/2);
        Monoid R = query(a, b, k*2+2, (l+r)/2, r);
        return merge(L, R);
    }

    Monoid query(int a, int b){
        return query(a, b, 0, 0, n);
    }
};

//########## 例 ##########
// Range update - min (AOJ DSL_2_F)
Monoid M0 = LLINF, L0 = LLINF;
Monoid merge(Monoid a, Monoid b){ return min(a, b); }
void updateNode(int k, Monoid x){ lazy[k] = x; }
void propagate(int k, int l, int r){ dat[k] = lazy[k]; }
LazySegtree<LL> seg(n+1, (1LL<<31)-1);

// Range update - sum (AOJ DSL_2_I)
Monoid M0 = 0, L0 = LLINF;
Monoid merge(Monoid a, Monoid b){ return a + b; }
void updateNode(int k, Monoid x){ lazy[k] = x; }
void propagate(int k, int l, int r){ dat[k] = lazy[k]*(r-l); }
LazySegtree<LL> seg(n+1, 0);

// Range add - min (AOJ DSL_2_H)
Monoid M0 = LLINF, L0 = 0;
Monoid merge(Monoid a, Monoid b){ return min(a, b); }
void updateNode(int k, Monoid x){ lazy[k] += x; }
void propagate(int k, int l, int r){ dat[k] += lazy[k]; }
LazySegtree<LL> seg(n+1, 0);

// Range add - sum (AOJ DSL_2_G)
Monoid M0 = 0, L0 = 0;
Monoid merge(Monoid a, Monoid b){ return a + b; }
void updateNode(int k, Monoid x){ lazy[k] += x; }
void propagate(int k, int l, int r){ dat[k] += lazy[k]*(r-l); }
LazySegtree<LL> seg(n+1, 0);
\end{lstlisting}

\newpage
\section{グラフ}
\minititle{Grid上でのBFS}
\begin{lstlisting}
int W, H;
vector<vector<char>> s;
vector<vector<int>> cost;
int bfs(){
    int dx[] = {0, 1, 0, -1}, dy[] = {1, 0, -1, 0};
    queue<pair<int, int>> que;
    que.push(make_pair(0, 0));
    cost[0][0] = 0;

    while(!que.empty()) {
        pair<int, int> p = que.front();
        que.pop();
        if(p == make_pair(H-1, W-1)){
            // ゴールに到達
            return cost[p.first][p.second];
        }
        for(int i = 0; i < 4; i++) {
            int ny = p.first + dy[i], nx = p.second + dx[i];
            if(nx < 0 || ny < 0 || nx >= W || ny >= H) continue;
            if(s[ny][nx] == '#') continue;
            if(cost[ny][nx] != -1) continue;

            cost[ny][nx] = cost[p.first][p.second] + 1;
            que.push(make_pair(ny, nx));
        }
    }
    return -1;
}
\end{lstlisting}

\minititle{Dijkstra法(単一始点最短経路)}
\begin{lstlisting}
struct edge{
    int to, cost;
};
using WeightedGraph = vector<vector<edge>>;

vector<int> dijkstra(WeightedGraph &G, int st){
    vector<int> dist(G.size(), INF);
    using pi = pair<int, int>;
    priority_queue<pi, vector<pi>, greater<pi>> que;
    dist[st] = 0;
    que.push(mp(dist[st], st));
    while(!que.empty()){
        int cost, idx;
        tie(cost, idx) = que.top(); que.pop();
        if(dist[idx] < cost) continue;
        for(auto &e: G[idx]){
            if(dist[e.to] <= cost+e.cost) continue;
            dist[e.to] = cost+e.cost;
            que.push(mp(dist[e.to], e.to));
        }
    }
    return dist;
}
\end{lstlisting}

\minititle{Bellman-Ford法(負路あり単一始点最短経路)}
\begin{lstlisting}
struct edge{
    int src, to, cost;
};
using Edges = vector<edge>;

vector<int> bellman_ford(Edges &E, int V, int st){
    vector<int> dist(V, INF);
    dist[st] = 0;
    rep(i, V-1){
        for(auto &e: E){
            if(dist[e.src] == INF) continue;
            dist[e.to] = min(dist[e.to], dist[e.src]+e.cost);
        }
    }
    for(auto &e: E){
        if(dist[e.src] == INF) continue;
        if(dist[e.to] > dist[e.src]+e.cost){
            // 負閉路が存在
            return vector<int>();
        }
    }
    return dist;
}
\end{lstlisting}

\minititle{Warshall-Floyd法(全点対間最短経路)}
\begin{lstlisting}
using Graph = vector<vector<int>>;

void warshall_floyd(Graph &G){
    int V = G.size();
    rep(k, V)rep(i, V)rep(j, V){
        if(G[i][k] == INF || G[k][j] == INF) continue;
        G[i][j] = min(G[i][j], G[i][k]+G[k][j]);
    }
    // G[i][i] < 0が存在 <=> 負閉路が存在
}
\end{lstlisting}

\minititle{Kruskal法(最小全域木)}
\begin{lstlisting}
// UnionFindが必要

struct edge{
    int src, to, cost;
};
using Edges = vector<edge>;

int kruskal(Edges &E, int V)
{
    sort(all(E), [](const edge &a, const edge &b)
    {
        return (a.cost < b.cost);
    });
    UnionFind tree(V);
    int res = 0;
    for(auto &e : E) {
        if(tree.unite(e.src, e.to)) res += e.cost;
    }
    return (res);
}
\end{lstlisting}

\minititle{トポロジカルソート}
\begin{lstlisting}
struct edge{
    int to, cost;
};
using WeightedGraph = vector<vector<edge>>;

vector<int> tsort(WeightedGraph &G){
    vector<int> tsorted;
    vector<int> used(G.size(), 0);
    bool f = false;
    function<void(int)> dfs = [&](int u){
        if(used[u] > 0){
            if(used[u] == 1) f = true;
            return;
        }
        used[u] = 1;
        for(auto &e : G[u]) dfs(e.to);
        used[u] = 2;
        tsorted.pb(u);
    };
    rep(i, G.size()) dfs(i);
    if(f){
        // 閉路が存在
        return vector<int>();
    }
    reverse(all(tsorted));
    return tsorted;
}
\end{lstlisting}

\minititle{Dinic法(最大流)}
\begin{lstlisting}
template<typename flow_t>
struct Dinic{
    const flow_t INF_flow_t = INF;   // WRITE HERE

    struct edge{
        int to;
        flow_t cap;
        int rev;
    };
    using WeightedGraph = vector<vector<edge>>;
    int V;
    WeightedGraph G;
    vector<int> itr, level;

    Dinic(int V) : V(V) { G.assign(V, vector<edge>()); }

    void add_edge(int from, int to, int cap) {
        G[from].push_back((edge){to, cap, (int)G[to].size()});
        G[to].push_back((edge){from, 0, (int)G[from].size()-1});
    }

    void bfs(int s) {
        level.assign(V, -1);
        queue<int> que;
        level[s] = 0;
        que.push(s);
        while (!que.empty()) {
            int v = que.front(); que.pop();
            for(auto &&e: G[v]){
                if (e.cap > 0 && level[e.to] < 0) {
                    level[e.to] = level[v] + 1;
                    que.push(e.to);
                }
            }
        }
    }

    flow_t dfs(int v, int t, flow_t f) {
        if(v == t) return f;
        for(int &i = itr[v]; i < (int)G[v].size(); i++) {
            edge &e = G[v][i];
            if (e.cap > 0 && level[v] < level[e.to]) {
                flow_t d = dfs(e.to, t, min(f, e.cap));
                if (d > 0) {
                    e.cap -= d;
                    G[e.to][e.rev].cap += d;
                    return d;
                }
            }
        }
        return 0;
    }

    flow_t max_flow(int s, int t) {
        flow_t res = 0, f;
        while(bfs(s), level[t] >= 0) {
            itr.assign(V, 0);
            while((f = dfs(s, t, INF_flow_t)) > 0) res += f;
        }
        return res;
    }
};

// 最小流量制限付き最大流
// 各辺に[lb, ub]の容量の辺を張る
template<typename flow_t>
struct DinicWithLowerBound{
    Dinic<flow_t> flow;
    int S, T;
    flow_t sum_lb;

    DinicWithLowerBound(int V) : flow(V+2), S(V), T(V+1), sum_lb(0) {}

    void add_edge(int from, int to, flow_t lb, flow_t ub) {
        flow.add_edge(from, to, ub-lb);
        flow.add_edge(S, to, lb);
        flow.add_edge(from, T, lb);
        sum_lb += lb;
    }

    flow_t max_flow(int s, int t) {
        auto a = flow.max_flow(S, T);
        auto b = flow.max_flow(s, T);
        auto c = flow.max_flow(S, t);
        auto d = flow.max_flow(s, t);
        return (b == c && a + b == sum_lb) ? b+d : -1;
    }
};
\end{lstlisting}

\minititle{二部マッチング}
\begin{lstlisting}
struct BipartiteMatching {
    using Graph = vector<vector<int>>;
    Graph G;
    vector<int> match, alive, used;
    int timestamp;

    BipartiteMatching(int n) : G(n), alive(n, 1),
                               used(n, 0), match(n, -1), timestamp(0) {}

    void add_edge(int u, int v) {
        G[u].push_back(v);
        G[v].push_back(u);
    }

    int dfs(int idx) {
        used[idx] = timestamp;
        for(auto &&to : G[idx]) {
            int w = match[to];
            if(alive[to] == 0) continue;
            if(w < 0 || (used[w] != timestamp && dfs(w))) {
                match[idx] = to;
                match[to] = idx;
                return 1;
            }
        }
        return 0;
    }

    int bipartite_matching() {
        int res = 0;
        for(int i = 0; i < G.size(); i++) {
            if(alive[i] == 0) continue;
            if(match[i] == -1) {
                ++timestamp;
                res += dfs(i);
            }
        }
        return res;
    }

     void output() {
        for(int i = 0; i < G.size(); i++) {
            if(i < match[i]) {
                cout << i << "-" << match[i] << endl;
            }
        }
    }
};
\end{lstlisting}

\section{木}
\minititle{木の直径}
\begin{lstlisting}
struct edge{
    int to, cost;
};
using WeightedGraph = vector<vector<edge>>;
using pi = pair<int, int>;

pi dfs(WeightedGraph &G, int idx, int src){
    pi res(0, idx);
    for(auto &e : G[idx]) {
        if(e.to == src) continue;
        pi cost = dfs(G, e.to, idx);
        cost.first += e.cost;
        res = max(res, cost);
    }
    return res;
}

int tree_diameter(WeightedGraph &G)
{
    auto far = dfs(G, 0, -1);
    auto res = dfs(G, far.second, -1);
    return (res.first);
}
\end{lstlisting}

\newpage
\section{数学}
\minititle{GCD・LCM}
\begin{lstlisting}
LL gcd(LL a, LL b){
    if(a < b) swap(a, b);
    if(b == 0) return a;
    return gcd(b, a%b);
}

LL lcm(LL a, LL b){
    return a*b/gcd(a,b);
}
\end{lstlisting}

\minititle{ModInt}
\begin{lstlisting}
template<int mod>
struct ModInt {
    int x;

    ModInt() : x(0) {}
    ModInt(long long y) : x(y >= 0 ? y % mod : (mod - (-y) % mod) % mod) {}

    ModInt &operator+=(const ModInt &p) {
        if((x += p.x) >= mod) x -= mod;
        return *this;
    }
    ModInt &operator-=(const ModInt &p) {
        if((x += mod - p.x) >= mod) x -= mod;
        return *this;
    }
    ModInt &operator*=(const ModInt &p) {
        x = (int)(1LL * x * p.x % mod);
        return *this;
    }
    ModInt &operator/=(const ModInt &p) {
        *this *= p.inverse();
        return *this;
    }

    ModInt operator-() const { return ModInt(-x); }
    ModInt operator+(const ModInt &p) const { return ModInt(*this) += p; }
    ModInt operator-(const ModInt &p) const { return ModInt(*this) -= p; }
    ModInt operator*(const ModInt &p) const { return ModInt(*this) *= p; }
    ModInt operator/(const ModInt &p) const { return ModInt(*this) /= p; }

    bool operator==(const ModInt &p) const { return x == p.x; }
    bool operator!=(const ModInt &p) const { return x != p.x; }

    ModInt inverse() const{
        int a = x, b = mod, u = 1, v = 0, t;
        while(b > 0) {
            t = a / b;
            a -= t * b;
            swap(a, b);
            u -= t * v;
            swap(u, v);
        }
        return ModInt(u);
    }

    ModInt pow(int e){
        long long a = 1, p = x;
        while(e > 0) {
            if(e%2 == 0) {p = (p*p) % mod; e /= 2;}
            else {a = (a*p) % mod; e--;}
        }
        return ModInt(a);
    }

    friend ostream &operator<<(ostream &os, const ModInt<mod> &p) {
        return os << p.x;
    }
    friend istream &operator>>(istream &is, ModInt<mod> &a) {
        long long x;
        is >> x;
        a = ModInt<mod>(x);
        return (is);
    }
};

const int mod = 1e9 + 7;
using modint = ModInt<mod>;
\end{lstlisting}

\minititle{FFT}
\begin{lstlisting}
// https://ei1333.github.io/luzhiled/snippets/math/fast-fourier-transform.html
namespace FastFourierTransform {
    using real = double;

    struct C {
        real x, y;
        C() : x(0), y(0) {}
        C(real x, real y) : x(x), y(y) {}
        inline C operator+(const C &c) const { return C(x + c.x, y + c.y); }
        inline C operator-(const C &c) const { return C(x - c.x, y - c.y); }
        inline C operator*(const C &c) const {
            return C(x * c.x - y * c.y, x * c.y + y * c.x);
        }
        inline C conj() const { return C(x, -y); }
    };

    const real PI = acosl(-1);
    int base = 1;
    vector< C > rts = { {0, 0},
                        {1, 0} };
    vector< int > rev = {0, 1};


    void ensure_base(int nbase) {
        if(nbase <= base) return;
        rev.resize(1 << nbase);
        rts.resize(1 << nbase);
        for(int i = 0; i < (1 << nbase); i++) {
            rev[i] = (rev[i >> 1] >> 1) + ((i & 1) << (nbase - 1));
        }
        while(base < nbase) {
        real angle = PI * 2.0 / (1 << (base + 1));
        for(int i = 1 << (base - 1); i < (1 << base); i++) {
            rts[i << 1] = rts[i];
            real angle_i = angle * (2 * i + 1 - (1 << base));
            rts[(i << 1) + 1] = C(cos(angle_i), sin(angle_i));
        }
            ++base;
        }
    }

    void fft(vector< C > &a, int n) {
        assert((n & (n - 1)) == 0);
        int zeros = __builtin_ctz(n);
        ensure_base(zeros);
        int shift = base - zeros;
        for(int i = 0; i < n; i++) {
            if(i < (rev[i] >> shift)) {
                swap(a[i], a[rev[i] >> shift]);
            }
        }
        for(int k = 1; k < n; k <<= 1) {
            for(int i = 0; i < n; i += 2 * k) {
                for(int j = 0; j < k; j++) {
                C z = a[i + j + k] * rts[j + k];
                a[i + j + k] = a[i + j] - z;
                a[i + j] = a[i + j] + z;
                }
            }
        }
    }

    vector< int64_t > multiply(const vector< int > &a, const vector< int > &b) {
        int need = (int) a.size() + (int) b.size() - 1;
        int nbase = 1;
        while((1 << nbase) < need) nbase++;
        ensure_base(nbase);
        int sz = 1 << nbase;
        vector< C > fa(sz);
        for(int i = 0; i < sz; i++) {
            int x = (i < (int) a.size() ? a[i] : 0);
            int y = (i < (int) b.size() ? b[i] : 0);
            fa[i] = C(x, y);
        }
        fft(fa, sz);
        C r(0, -0.25 / (sz >> 1)), s(0, 1), t(0.5, 0);
        for(int i = 0; i <= (sz >> 1); i++) {
            int j = (sz - i) & (sz - 1);
            C z = (fa[j] * fa[j] - (fa[i] * fa[i]).conj()) * r;
            fa[j] = (fa[i] * fa[i] - (fa[j] * fa[j]).conj()) * r;
            fa[i] = z;
        }
        for(int i = 0; i < (sz >> 1); i++) {
            C A0 = (fa[i] + fa[i + (sz >> 1)]) * t;
            C A1 = (fa[i] - fa[i + (sz >> 1)]) * t * rts[(sz >> 1) + i];
            fa[i] = A0 + A1 * s;
        }
        fft(fa, sz >> 1);
        vector< int64_t > ret(need);
        for(int i = 0; i < need; i++) {
            ret[i] = llround(i & 1 ? fa[i >> 1].y : fa[i >> 1].x);
        }
        return ret;
    }
};
// auto C = FastFourierTransform::multiply(A, B);
\end{lstlisting}

\minititle{Lagrange補間}
\begin{lstlisting}
template< typename T >
T lagrange_polynomial(const vector< T > &y, int64_t t) {
  int N = y.size() - 1;
  Combination< T > comb(N);
  if(t <= N) return y[t];
  T ret(0);
  vector< T > dp(N + 1, 1), pd(N + 1, 1);
  for(int i = 0; i < N; i++) dp[i + 1] = dp[i] * (t - i);
  for(int i = N; i > 0; i--) pd[i - 1] = pd[i] * (t - i);
  for(int i = 0; i <= N; i++) {
    T tmp = y[i] * dp[i] * pd[i] * comb.rfact(i) * comb.rfact(N - i);
    if((N - i) & 1) ret -= tmp;
    else ret += tmp;
  }
  return ret;
}

int main() {
  int N, T;
  cin >> N;
  vector< modint > A(N + 1);
  cin >> A;
  cin >> T;
  cout << lagrange_polynomial(A, T) << endl;
}
\end{lstlisting}

\section{文字列}
\minititle{KMP法}

文字列$S[0,i-1]$の接頭辞と接尾辞が最大何文字一致しているかを記録した配列を $O(|S|)$で構築する
\begin{lstlisting}
vector<int> A(s.size()+1);
A[0] = -1;
int j = -1;
for (int i = 0; i < s.size(); i++) {
    while (j >= 0 && s[i] != s[j]) j = A[j];
    j++;
    A[i+1] = j;
}
\end{lstlisting}

\minititle{ローリングハッシュ}
\begin{lstlisting}
struct RollingHash {
    using T = long long;
    using hash_type = pair<T, T>;
    const T mod1 = 1000000007, mod2 = 1000000009;
    T base1, base2;
    vector<T> hash1, hash2, pow1, pow2;

    RollingHash(const string &s){
        int n = s.size();
        std::random_device rnd;
        std::mt19937 mt(rnd());
        base1 = mt()%(mod1-2)+2;
        base2 = mt()%(mod2-2)+2;

        hash1.assign(n+1,0);
        hash2.assign(n+1,0);
        pow1.assign(n+1,1);
        pow2.assign(n+1,1);
        for(int i=0; i<n; i++){
            hash1[i+1] = (hash1[i]*base1+s[i])%mod1;
            hash2[i+1] = (hash2[i]*base2+s[i])%mod2;
            pow1[i+1]=pow1[i]*base1%mod1;
            pow2[i+1]=pow2[i]*base2%mod2;
        }
    }

    // [l, r)
    hash_type get(int l,int r){
        T res1 = ((hash1[r] - hash1[l]*pow1[r-l]) % mod1 + mod1) % mod1;
        T res2 = ((hash2[r] - hash2[l]*pow2[r-l]) % mod2 + mod2) % mod2;
        return make_pair(res1, res2);
    }

    hash_type concat(hash_type h1, hash_type h2, int h2len) {
        T res1 = (h1.first*pow1[h2len] + h2.first) % mod1;
        T res2 = (h1.second*pow2[h2len] + h2.second) % mod2;
        return make_pair(res1, res2);
    }
    // LCPは長さで二分探索する
};
\end{lstlisting}


\section{テクニック}
\minititle{座標圧縮}
\begin{lstlisting}
vector<int> unzip = a;
map<int, int> zip;
sort(all(unzip));
unzip.erase(unique(all(unzip)), unzip.end());
for(int i=0; i<unzip.size(); i++) zip[unzip[i]] = i;
\end{lstlisting}

\newpage
\minititle{スライド最小値}

$[i-k, i]$の最小値を格納したvectorを返す
\begin{lstlisting}
vector<int> slide_min(vector<int> &a, int k){
    deque<int> deq;
    vector<int> b;
    rep(i, a.size()){
        // maxはここの不等号の向きを変える
        while(!deq.empty() && a[deq.back()] >= a[i]) deq.pop_back();
        deq.push_back(i);
        b.push_back(a[deq.front()]);
        if(i-k+1 >= 0 && deq.front() == i-k+1) deq.pop_front();
    }
    return b;
}
\end{lstlisting}


\end{document}